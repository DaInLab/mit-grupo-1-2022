% Options for packages loaded elsewhere
\PassOptionsToPackage{unicode}{hyperref}
\PassOptionsToPackage{hyphens}{url}
%
\documentclass[
]{article}
\usepackage{amsmath,amssymb}
\usepackage{lmodern}
\usepackage{iftex}
\ifPDFTeX
  \usepackage[T1]{fontenc}
  \usepackage[utf8]{inputenc}
  \usepackage{textcomp} % provide euro and other symbols
\else % if luatex or xetex
  \usepackage{unicode-math}
  \defaultfontfeatures{Scale=MatchLowercase}
  \defaultfontfeatures[\rmfamily]{Ligatures=TeX,Scale=1}
\fi
% Use upquote if available, for straight quotes in verbatim environments
\IfFileExists{upquote.sty}{\usepackage{upquote}}{}
\IfFileExists{microtype.sty}{% use microtype if available
  \usepackage[]{microtype}
  \UseMicrotypeSet[protrusion]{basicmath} % disable protrusion for tt fonts
}{}
\makeatletter
\@ifundefined{KOMAClassName}{% if non-KOMA class
  \IfFileExists{parskip.sty}{%
    \usepackage{parskip}
  }{% else
    \setlength{\parindent}{0pt}
    \setlength{\parskip}{6pt plus 2pt minus 1pt}}
}{% if KOMA class
  \KOMAoptions{parskip=half}}
\makeatother
\usepackage{xcolor}
\usepackage[margin=1in]{geometry}
\usepackage{graphicx}
\makeatletter
\def\maxwidth{\ifdim\Gin@nat@width>\linewidth\linewidth\else\Gin@nat@width\fi}
\def\maxheight{\ifdim\Gin@nat@height>\textheight\textheight\else\Gin@nat@height\fi}
\makeatother
% Scale images if necessary, so that they will not overflow the page
% margins by default, and it is still possible to overwrite the defaults
% using explicit options in \includegraphics[width, height, ...]{}
\setkeys{Gin}{width=\maxwidth,height=\maxheight,keepaspectratio}
% Set default figure placement to htbp
\makeatletter
\def\fps@figure{htbp}
\makeatother
\setlength{\emergencystretch}{3em} % prevent overfull lines
\providecommand{\tightlist}{%
  \setlength{\itemsep}{0pt}\setlength{\parskip}{0pt}}
\setcounter{secnumdepth}{-\maxdimen} % remove section numbering
\ifLuaTeX
  \usepackage{selnolig}  % disable illegal ligatures
\fi
\IfFileExists{bookmark.sty}{\usepackage{bookmark}}{\usepackage{hyperref}}
\IfFileExists{xurl.sty}{\usepackage{xurl}}{} % add URL line breaks if available
\urlstyle{same} % disable monospaced font for URLs
\hypersetup{
  pdftitle={Impacto da COVID-19 nos Estudantes Universitários no Brasil},
  hidelinks,
  pdfcreator={LaTeX via pandoc}}

\title{Impacto da COVID-19 nos Estudantes Universitários no Brasil}
\usepackage{etoolbox}
\makeatletter
\providecommand{\subtitle}[1]{% add subtitle to \maketitle
  \apptocmd{\@title}{\par {\large #1 \par}}{}{}
}
\makeatother
\subtitle{Trabalho Final da Disciplina Ciência de Dados - 2022}
\author{true \and true \and true \and true}
\date{31 de janeiro de 2023}

\begin{document}
\maketitle
\begin{abstract}
Tendo em vista os fatores causados pela situação pandêmica mundial da
COVID-19, a proposta desta pesquisa é a de lançar alguma luz neste
âmbito e coletar dados exploratórios. O estudo proposto visa compreender
como os estudantes universitários estão vivenciando a pandemia e de que
forma se comportam frente a esta nova realidade.\\
Portanto, os objetivos dessa pesquisa exploratória-descritiva são de
investigar: a) como os alunos vivenciaram a pandemia da COVID-19; b) de
que forma se comportaram frente as restrições impostas pelos riscos de
contágio; c) quais suas considerações a respeito das estratégias que
foram adotadas pelas instituições superiores; e d) como estes fatores
influenciaram suas vidas.
\end{abstract}

\hypertarget{introduuxe7uxe3o}{%
\subsection{1. Introdução}\label{introduuxe7uxe3o}}

Em março de 2020, a Organização Mundial das Nações Unidas (ONU) declarou
oficialmente a pandemia da Covid-19 (UNA-SUS, 2020), doença infecciosa
causada pelo vírus SARS-CoV-2, detectado pela primeira vez em dezembro
de 2019, na China. No Brasil, foi dada a confirmação do primeiro caso em
fevereiro de 2020, tendo o número de infectados e óbitos se multiplicado
rapidamente posteriormente.

Medidas de prevenção foram implementadas em todas as regiões do país
para coibir a disseminação do vírus, tendo em vista a necessidade de
preparar as unidades de pronto atendimento (UPA) e hospitais para
receberem pacientes em larga escala em um curto período de tempo, uma
vez que a contaminação entre os seres humanos é considerada alta.

No Brasil, a primeira vacina a ser utilizada foi a CoronaVac,
desenvolvida pelo Instituto Butantã em parceria com a fabricante chinesa
de medicamentos Sinovac Biontech. A vacinação da população deu-se início
em janeiro de 2021. Desde então, 80,3\% da população já completou o
calendário vacinal (UOL, 2023) com todas as doses recomendadas, porém,
órgãos de saúde pública e privada ainda trabalham para que a cobertura
vacinal seja ainda maior, tendo em vista que a vacina deve ser
reforçada, em média, a cada 4 meses.

Seu impacto global afetou diversos setores como saúde, política,
economia e educação. Considerando-se os fatores causados pela COVID-19,
a proposta desta pesquisa exploratória-descritiva é compreender como os
estudantes universitários vivenciam a pandemia. Buscou-se investigar
também de que forma se comportam frente às restrições impostas pelos
riscos de contágio, quais suas considerações a respeito das estratégias
que foram adotadas pelas instituições superiores, e como estes fatores
influenciaram suas vidas.

\hypertarget{fundamentauxe7uxe3o-teuxf3rica}{%
\subsection{2. Fundamentação
Teórica}\label{fundamentauxe7uxe3o-teuxf3rica}}

A COVID-19 é uma síndrome respiratória aguda grave (SRAG) infecciosa
causada pelo coronavírus, cujo agente etiológico é o SARS-CoV-2. Sua
particularidade está na rapidez com que se manifesta entre seres
humanos, levando a alta contaminação e elevação do número de casos
(Campos, Mônica Rodrigues et al.~2020).. Inicialmente, os sintomas mais
comuns eram febre, tosse seca, perda dos sentidos, olfato e paladar, e,
em casos mais moderados/graves, falta de ar. Contudo, a doença apresenta
manifestações diferentes a depender do indivíduo. Desde então, variadas
mutações surgiram em diferentes partes do mundo, algumas, inclusive,
associadas a uma maior transmissibilidade e virulência.

\begin{quote}
Por isso, medidas de proteção como usar máscaras e higienizar as mãos
com sabão e álcool em gel, evitar aglomerações e manter o distanciamento
social, além de completar o esquema vacinal contra a Covid-19, são
iniciativas que funcionam contra todas variantes da Covid-19. (BUTANTAN,
2021).
\end{quote}

Além disso, as medidas preventivas incluíram, inicialmente, o isolamento
social, quando as autoridades recomendaram que a população permanecesse
em casa, evitando assim aglomerações e, consequentemente, a transmissão
do vírus. Nesse período, os estados e municípios adotaram diferentes
medidas de isolamento social, com diferentes níveis de restrição dado o
número de infectados.

No auge da pandemia, antes da criação de uma vacina, foi aplicado o
Lockdown (ou quarentena rígida) em determinadas localidades por um
determinado período de tempo, restringindo a circulação de pessoas e o
fechamento de comércios e serviços considerados não essenciais. Por
último, seguiu-se a vacinação, com uma campanha conduzida de forma
descentralizada, com cada Estado e município sendo responsáveis por
adquirir e distribuir as vacinas (VEJA, 2022). Além de tais medidas,
também puderam adotar outras adicionais de acordo com a situação local,
como fechamento de escolas, restrições de circulação noturna, entre
outras.

As principais críticas às decisões tomadas pelos governos, tanto na
esfera federal, estadual ou municipal do Brasil, se dão pela falta de
planejamento. O país sofreu com a ausência de um plano efetivo na
coordenação e distribuição de vacinas, tendo dificuldades principalmente
quanto à distribuição, com desequilíbrios regionais e problemas de
logística. A falta de apoio econômico e social para as pessoas afetadas
pela pandemia também foi um fator de grande impacto (CEPAL, 2021), pois
muitos foram afetados economicamente pela pandemia e pela necessidade de
isolamento social.

A falta de transparência e de informações precisas sobre a pandemia,
fosse pela real situação e as medidas governamentais adotadas tem sido
criticada, assim como a falta de informações consistentes e atualizadas.
E uma das principais causas na demora da compra de vacinas, e
impedimento de imunização no país, é a política de negacionismo, onde
governadores e líderes políticos minimizaram a gravidade da pandemia (O
Globo, 2021) e se recusaram a adotar medidas recomendadas pelas
autoridades de saúde mundiais, fator extremamente criticado por
especialistas (FIOCRUZ,2021), pois pode levou a uma disseminação
descontrolada do vírus e uma sobrecarga nos sistemas de saúde.

\hypertarget{metodologia}{%
\subsection{3. Metodologia}\label{metodologia}}

Para a metodologia foi levado em consideração que se fez necessário uma
abordagem quali-quantitativa, por envolver aspectos opinativos e
informações a fim de identificar as principais dificuldades enfrentadas.
Tal abordagem foi escolhida dada a coleta de dados, e com a corroboração
de outras fontes de informação para a análise, a fim de enriquecer a
pesquisa realizada.

\begin{quote}
Toda a ciência é qualitativa, no sentido que pretende estabelecer uma
qualidade a um objeto de estudo ao reproduzi-lo ou reconstruí-lo, ao
explicá-lo ou compreendê-lo. A quantidade em si mesma nada representa se
não se relaciona com determinada qualidade; as cifras e os dados não
falam sozinhos, requerem uma interpretação que alude a uma teoria, à
afirmação ou à negação de uma idéia. (\ldots) técnicas quantitativas de
levantamentos (surveys) que serão processados estatisticamente ou com
histórias de vida que serão analisadas qualitativamente.
(Briceño-Léon,2003)
\end{quote}

Na intenção de identificar as principais dificuldades pelas instituições
de ensino superior, foram adotados como instrumento de coleta de dados o
uso de um questionários online cadastrados no Google Formulários.

Este relatório representa a análise sobre como a pandemia de Coronavírus
(Covid-19) afetou a vida e cotidiano dos estudantes universitários. O
questionário ficou disponível de 22 de abril de 2021 e 06 de dezembro de
2022, contendo 50 questões. A pesquisa contou com a participação de 52
integrantes, sendo que mais de 48\% se declaram do gênero masculino,
46\% feminino, 4\% transgênero/transexual e 2\% homem gay. A amostra foi
composta ainda por 77\% de alunos da UNESP e 23\% de demais
instituições.

\hypertarget{anuxe1lise-exploratuxf3ria-de-dados}{%
\subsection{4. Análise Exploratória de
Dados}\label{anuxe1lise-exploratuxf3ria-de-dados}}

A Análise Exploratória de Dados, do inglês, \emph{Exploratory Data
Analysis (EDA)}, é uma abordagem utilizada por Cientistas de Dados para
analisar e investigar dados. A partir dessa análise é possível ter uma
visão panorâmica dos dados, a fim de obter sentido e extrair
conhecimentos. Nessa etapa, ainda não é possível compreender o que os
dados têm para dizer, mas é possível gerar insights para obter respostas
para as perguntas, além de coletar informações que podem ser usadas para
alimentar os modelos de \emph{machine learning}.

\hypertarget{preparauxe7uxe3o-e-compreensuxe3o-do-conteuxfado-de-dados}{%
\subsubsection{4.1 Preparação e compreensão do conteúdo de
dados}\label{preparauxe7uxe3o-e-compreensuxe3o-do-conteuxfado-de-dados}}

\hypertarget{importauxe7uxe3o-dos-dados}{%
\subsubsection{4.2 Importação dos
dados}\label{importauxe7uxe3o-dos-dados}}

\hypertarget{visualizauxe7uxe3o-dos-dados}{%
\subsubsection{4.3 Visualização dos
dados}\label{visualizauxe7uxe3o-dos-dados}}

Para o perfil dos entrevistados foram analisadas as respostas obtidas
pelo questionário, no período disponível.

Portanto, neste tópico do artigo serão mostrados e discutidos os
gráficos resultantes da AED realizada.

\begin{verbatim}
## Warning: package 'readxl' was built under R version 4.2.2
\end{verbatim}

\includegraphics{covid19-i-hes-grupo-1_files/figure-latex/unnamed-chunk-1-1.pdf}

Como pode ser observado no \textbf{Gráfico 1}, A faixa etária com maior
número de entrevistados, correspondendo a 35\%, é a faixa etária de 17 a
24 anos. As faixas de 22 a 26 anos e 37 a 41 anos obtiveram 13\% cada
uma, 10\% dos respondentes estão na faixa de 42 a 46 anos, de 47 a 51
anos e de 52 a 56 anos tiveram ambas 8\%, 6\% eram da faixa de 27 a 31
anos, 6\% de 32 a 36 anos, e apenas 2\% se enquadraram na faixa etária
de 57 a 61 anos.

\includegraphics{covid19-i-hes-grupo-1_files/figure-latex/unnamed-chunk-2-1.pdf}

Observando-se o \textbf{Gráfico 2}, constata-se que, na amostra
utilizada nesta análise exploratória, 54\% dos respondentes são
solteiros(as), 31\% casados(as), 10\% estão em uma união estável, 4\%
divorciado(a) e 1\% viúvo(a)

\includegraphics{covid19-i-hes-grupo-1_files/figure-latex/unnamed-chunk-3-1.pdf}

Ainda caracterizando o perfil dos entrevistados no \textbf{Gráfico 3}, o
nível de escolaridade predominante é a graduação, com 21 respondentes
(40\%), seguido pelo doutorado com 15 (29\%) e mestrado com 13
respondentes. Enquanto isso, o nível de Pós-doutorado,
MBA/Especialização e ensino técnico obtiveram 1 entrevistado cada (2\%).
Desses, 45 são de instituição pública, 6 de privada, e 1 de autarquia
municipal.

\includegraphics{covid19-i-hes-grupo-1_files/figure-latex/unnamed-chunk-4-1.pdf}

Ao analisar as condições financeiras dos participantes no
\textbf{Gráfico 4}, foi possível identificar que, dos 52 respondentes,
30 estavam empregados, em diversos segmentos, 10 eram bolsistas e 2
estagiários , 7 eram dependentes e contavam com ajuda dos pais , 2
estavam desempregados, e 1 aposentado. Sendo assim, dos 52 respondentes,
15\% tiveram alguma ajuda financeira de sua instituição educacional ou
de outra organização durante a pandemia, e 85\% não obteve ajuda.

\includegraphics{covid19-i-hes-grupo-1_files/figure-latex/unnamed-chunk-5-1.pdf}

Quanto à vacinação, apresentada no \textbf{Gráfico 5}, não houve nenhum
participante não vacinado contra a COVID-19. Em sua maioria, foram
tomadas as duas doses, ou dose única, e reforço. Outro fator importante
a ser levado em consideração, devido a pandemia, é a questão de
locomoção - impedidas pelas restrições e lockdown - e o fato de que a
maior parte dos respondentes não eram residentes da cidade em que
estudavam, sendo apenas 18 entrevistados locais e 34 de outras cidades.
Destes, 90\% estavam na sua residência permanente durante a pandemia e
10\% não.

\includegraphics{covid19-i-hes-grupo-1_files/figure-latex/unnamed-chunk-6-1.pdf}

Em relação ao \textbf{Gráfico 6}, do total de respondentes, 69\% moravam
com a família, seguido de sozinho (25\%), e em último ficaram os que
residiam em república ou com colega de quarto (6\%).

\includegraphics{covid19-i-hes-grupo-1_files/figure-latex/unnamed-chunk-7-1.pdf}

De acordo com os entrevistados, conforme mostra o \textbf{Gráfico 7},
eles entenderam que as universidades fecharam e tomaram a decisão de
utilizar ferramentas online de forma oportuna e prudente (77\%), muito
rapidamente 10\%, e lentamente, apenas 13\%. Com relação à decisão de
fechar o campus e de utilizar ferramentas online para as aulas, os
respondentes afirmaram que as decisões na sua instituição foram tomadas,
em sua maioria, de forma prudente e oportuna (77\%). Sendo que apenas
2\% informaram que a universidade não migrou para as aulas virtuais
durante a pandemia.

\includegraphics{covid19-i-hes-grupo-1_files/figure-latex/unnamed-chunk-8-1.pdf}

Partindo da necessidade em utilizar a internet para as aulas, como pode
se observar no \textbf{Gráfico 8}, foi questionado quanto a alteração na
acessibilidade à Internet, antes e durante a pandemia da Covid-19, de
acordo com 65\% não houve nenhuma mudança, para 10\% piorou e 4\% piorou
muito. Numa outra visão, 17\% sentiram uma melhora e 2\% afirmaram que
melhorou muito. Apenas 2\% não souberam responder.

\includegraphics{covid19-i-hes-grupo-1_files/figure-latex/unnamed-chunk-9-1.pdf}

No \textbf{Gráfico 9}, ao abordarmos sobre o corpo docente das
universidades, a maioria dos respondentes sentiu que o acesso aos
professora piorou (52\%), seguido dos que acharam que foi mais ou menos
o mesmo (27\%) e para uma parcela menor (15\%) melhorou, 8\% não
souberam responder ou não se aplicava.

\includegraphics{covid19-i-hes-grupo-1_files/figure-latex/unnamed-chunk-10-1.pdf}

No \textbf{Gráfico 10}, com relação às formas de acesso aos recursos de
infraestrutura oferecidos pela sua instituição (biblioteca, coordenação,
orientação de assuntos acadêmicos, matrícula, etc.), durante a pandemia
do COVID-19 para 33\% dos entrevistados não mudou, 40\% sentiram que
piorou, para 13\% melhorou e 13\% não souberam dizer.

\includegraphics{covid19-i-hes-grupo-1_files/figure-latex/unnamed-chunk-11-1.pdf}

Após a vacinação e controle da pandemia, pode se observar no
\textbf{Gráfico 11} que, 60\% dos entrevistados afirmaram que as
universidades já retomaram as atividades de forma presencial no campus,
29\% em parte, 8\% não retomaram e 2\% não sabiam informar.

\includegraphics{covid19-i-hes-grupo-1_files/figure-latex/unnamed-chunk-12-1.pdf}

Com as restrições sociais introduzidas durante o período de isolamento
social, é importante entender como a vida financeira foi afetada. Sendo
assim, em relação aos gastos durante o período da pandemia, assim como
mostra no \textbf{Gráfico 12}, 40\% alega que as despesas não sofreram
alteração. 33\% teve um aumento e 27\% tiveram uma redução.

\includegraphics{covid19-i-hes-grupo-1_files/figure-latex/unnamed-chunk-13-1.pdf}

Apesar da maioria dos entrevistados não terem sofrido com o aumento de
despesas, no \textbf{Gráfico 13}, 44\% alega que sua renda financeira
continuou mais ou menos a mesma coisa. 25\% observaram uma queda no
poder de compra, 6\% tiveram um aumento e 2\% não quiseram e/ou puderam
opinar.

Dentre os gastos citados pelos respondentes pós pandemia, o maior foi
alimentação (25\%), seguido pelo transporte urbano e deslocamento.
Levando em consideração que, em sua maioria, os entrevistados tiveram
que retomar pelo menos parcialmente ou híbrida para seu trabalho e
faculdade, justifica ser um dos itens mais marcados.

\includegraphics{covid19-i-hes-grupo-1_files/figure-latex/unnamed-chunk-14-1.pdf}

Ainda que muitas universidades públicas ofereçam auxílio financeiro para
que estudantes possam morar no campus (ou próximo de) para estudarem,
durante o período de pandemia, constata-se no \textbf{Gráfico 14}, que
44\% dos entrevistados não tiveram nenhum respaldo por parte da
instituição.

\includegraphics{covid19-i-hes-grupo-1_files/figure-latex/unnamed-chunk-15-1.pdf}

De acordo com o \textbf{Gráfico 15}, 98\% dos entrevistados migraram
para o ensino remoto.

\includegraphics{covid19-i-hes-grupo-1_files/figure-latex/unnamed-chunk-16-1.pdf}

Com a migração para o ensino virtual remoto e a mudança na forma de
aprendizado, 38\% dos respondentes no \textbf{Gráfico 16} considera que
seu desempenho continuou o mesmo se comparado ao presencial. 31\% teve
seu rendimento comprometido, 27\% continuaram com a mesma performance e
2\% não souberam opinar.

\includegraphics{covid19-i-hes-grupo-1_files/figure-latex/unnamed-chunk-17-1.pdf}

De modo geral, o \textbf{Gráfico 17} mostra que o acesso a internet não
mudou para 65\% dos estudantes, porém, houve melhora para 19\%, piora
para 13\% e 2\% não souberam opinar.

\includegraphics{covid19-i-hes-grupo-1_files/figure-latex/unnamed-chunk-18-1.pdf}

Por fim, o \textbf{Gráfico 18} mostra que o acesso a toda a estrutura
para dar prosseguimento aos estudos piorou, na opinião de 40\% dos
estudantes. 33\% alegam que não houveram mudanças. Houve empate entre
aqueles que alegam melhorias e não sabem responder, ambos os lados
representam 13\%.

\hypertarget{resultados-e-discussuxe3o}{%
\subsection{5. Resultados e Discussão}\label{resultados-e-discussuxe3o}}

Após a análise dos dados oriundos do questionário, foi observado que as
respostas recebidas não estão totalmente distantes da realidade
brasileira após período crítico de pandemia. Porém, deve-se ressaltar
alguns tópicos que chamam a atenção por revelarem informações sobre
dificuldades que, até então, pareciam representar uma parcela distante
da população comparada à bolha social de cada indivíduo.

Como primeiro exemplo, deve-se citar os 2\% universitários que não
migraram para o ensino virtual. Esse valor ainda foi superior que o
identificado em escolas públicas e particulares, segundo pesquisa
divulgada pela Agência Senado:

\begin{quote}
No ensino privado, 70,9\% das escolas ficaram fechadas no ano passado. O
número é consideravelmente menor que o da rede pública: 98,4\% das
escolas federais, 97,5\% das municipais e 85,9\% das estaduais. (Agência
Senado)
\end{quote}

Contudo, de acordo com o Presidente da Comissão de Educação da Alerj
durante a pandemia, Flavio Serafini (Psol), a rede estadual não
conseguia apresentar (na época) uma solução que garantisse o direito ao
ensino durante a pandemia.

\begin{quote}
Mais de um terço dos alunos sequer acessou o aplicativo do estado, e a
média de uso diário é inferior a 10\% do total de estudantes na maioria
dos dias. Isso mostra que o que foi desenvolvido até agora é muito
limitado. Faltou uma política de inclusão digital mais estruturante.''
(EXTRA, 2021)
\end{quote}

Isso fez com que a qualidade da aprendizagem caísse e o déficit
educacional aumentasse, agravando ainda mais a desigualdade.

A crise financeira iniciada em 2014 foi causada por um conjunto de
choques de oferta e demanda, obrigando gestores públicos a adotarem
instrumentos políticos para atenuar seus efeitos (Mariano, 2016), porém
a tão temida crise da pandemia não teve quaisquer indícios ou sinais
semelhantes ao anterior. De acordo com o relatório do Banco Mundial, a
recessão decorrente da pandemia atingiu seu ápice, em número de países
atingidos, nos últimos 120 anos (PODER 360, 2022).

\begin{quote}
``O Brasil desembolsou 15\% do PIC (\ldots) para conter os efeitos da
covid no 1 ano de pandemia. (\ldots) O endividamento dos países tende a
se agravar, disse Ramalho, por outra consequência global da pandemia: a
alta da inflação.'' (PODER 360, 2022)
\end{quote}

De acordo com a fonte, o Banco Mundial ainda estima que 76 milhões de
pessoas entraram em 2020 para a extrema pobreza. O crescimento da
desigualdade não foi causado apenas por conta da pandemia, contudo os
mais vulneráveis ficaram excluídos até de medidas como o Auxílio
Emergencial, já que muitos não têm acesso à internet. Segundo o IBGE, em
2021, 92,3\% dos domicílios urbanos brasileiros têm acesso à Internet
(PNAD, 2022). Entretanto, a desigualdade também se apresenta nesses
casos, já que 100\% dos lares da classe A têm ao menos um computador, e
apenas 13\% dos de classe D e E.

Paralelamente, é importante destacar que boa parte dos entrevistados
sofreram consequências mais brandas no âmbito educacional. Apesar de
enfrentarem as mesmas dificuldades, como por exemplo, a falta de acesso
à estrutura universitária em determinadas situações, de modo geral
pode-se concluir que não sofreram impactos grandiosos que resultam no
rompimento das atividades acadêmicas ou sua conclusão.

\hypertarget{conclusuxe3o}{%
\subsection{6. Conclusão}\label{conclusuxe3o}}

Diante dos resultados apresentados, conclui-se que a maior parte dos
estudantes universitários desta pesquisa conseguiu passar pelo período
problemático da pandemia sem grandes mudanças na sua vida acadêmica.

As universidades em que estão matriculados migraram para o ensino remoto
e adaptando-se às necessidades para atender os docentes, inclusive no
quesito infraestrutura, mesmo boa parte relatando algumas dificuldades
em acessá-la sem grandes problemas.

No quesito financeiro, ao contrário das expectativas de refletir o
cenário nacional, os resultados mostraram que boa parte não sofreu
grandes impactos e puderam continuar com sua rotina sem alardes. Um
ponto interessante a ser citado uma vez que, mesmo possuindo fonte de
renda, alguns receberam auxílio para complementar.

Quanto ao desempenho durante o período de ensino remoto, ainda que
alguns sentiram maiores dificuldades e até mesmo uma queda em sua
performance, conta-se um desempenho positivo entre a maioria que manteve
a mesma produtividade.

Uma das principais limitações deste estudo é o fato de a amostra engloba
basicamente estudantes de uma única universidade pública no Brasil, ou
seja, os resultados podem não ser representativos de toda a população
universitária em nível nacional ou até mesmo regional. No entanto, a
taxa está dentro da faixa esperada de respostas, pelos autores, usando
pesquisas on-line com estudantes universitários.

Considerando que não usamos uma amostragem aleatória e que a prevalência
de problemas financeiros, logísticos de isolamento, de desempenho
estudantil, e satisfação quanto às medidas adotadas pelas universidades,
podem variar de acordo com o momento em que a avaliação foi feita, e
concepção pessoal. Por conta das diferenças populacionais, e
instrumentos usados, comparações entre outros levantamentos e pesquisas
devem ser feitas com cautela.

\hypertarget{referuxeancias-bibliogruxe1ficas}{%
\subsection{Referências
Bibliográficas}\label{referuxeancias-bibliogruxe1ficas}}

BUTANTAN. \textbf{Por que acontecem mutações do SARS-CoV-2 e quais as
diferenças entre cada uma das variantes.} 2021 . Disponível em:
\url{https://butantan.gov.br/noticias/por-que-acontecem-mutacoes-do-sars-cov-2-e-quais-as-diferencas-entre-cada-uma-das-variantes}

Briceño-León, R. (2003). \textbf{Quatro modelos de integração de
técnicas qualitativas e quantitativas de investigação nas ciências
sociais.} O Clássico e o novo--tendências, objetos e abordagens em
ciências sociais e saúde, 157-186. Disponível em:
\url{https://books.scielo.org/id/d5t55/pdf/goldenberg-9788575412510.pdf\#page=157}

Campos, Mônica Rodrigues et al.~\textbf{Carga de doença da COVID-19 e de
suas complicações agudas e crônicas: reflexões sobre a mensuração (DALY)
e perspectivas no Sistema Único de Saúde.} Cadernos de Saúde Pública
{[}online{]}. v. 36, n.~11 {[}Acessado 23 Janeiro 2023{]} , e00148920.
Disponível em: \url{https://doi.org/10.1590/0102-311X00148920}. ISSN
1678-4464. \url{https://doi.org/10.1590/0102-311X00148920}.

CEPAL. \textbf{Pandemia provoca aumento nos níveis de pobreza sem
precedentes nas últimas décadas e tem um forte impacto na desigualdade e
no emprego.} 2021. Disponível em:
\url{https://www.cepal.org/pt-br/comunicados/pandemia-provoca-aumento-niveis-pobreza-sem-precedentes-ultimas-decadas-tem-forte}

EXTRA. \textbf{Migração para as escolas públicas cresce 30\% na
pandemia, e rede privada perde 50 mil alunos.} 2021. Disponível em:
\url{https://extra.globo.com/noticias/rio/migracao-para-as-escolas-publicas-cresce-30-na-pandemia-rede-privada-perde-50-mil-alunos-25138212.html}

FIOCRUZ. 2021. \textbf{Aula inaugural analisa consequências das decisões
brasileiras no enfrentamento à pandemia.} Disponível em:
\url{https://informe.ensp.fiocruz.br/noticias/51262}

IBGE. PNAD Contínua - \textbf{Módulo de Tecnologia de Informação e
Comunicação} (TIC) 2021. Disponível em:
\url{https://www.ibge.gov.br/estatisticas/multidominio/ciencia-tecnologia-e-inovacao/17270-pnad-continua.html?=\&t=resultados}

MARIANO, Jefferson, Barcellos, Lívia I. (2017). \textbf{Estratégias de
gestão dos municípios em cenário de crise socioeconômica.} Geografia e
Pesquisa, 11(2).

O GLOBO. \textbf{Por que o negacionismo atrapalha o combate à Covid?.}
2021. Disponível em:
\url{https://oglobo.globo.com/podcast/por-que-negacionismo-atrapalha-combate-covid-1-24931788}

PODER 360. \textbf{Pandemia causou recessão mais ampla que as guerras
mundiais.} 2022. Disponível em:
\url{https://www.poder360.com.br/economia/pandemia-causou-recessao-mais-ampla-que-as-guerras-mundiais/}

UOL. \textbf{Covid: 172,6 milhões de brasileiros completam vacinação,
80,3\% da população.} Disponível em:
\url{https://noticias.uol.com.br/saude/ultimas-noticias/redacao/2023/01/11/vacinacao-covid-19-coronavirus-11-de-janeiro.htm?cmpid=copiaecola}.
Acesso em: 12 de jan. de 2023.

UNA-SUS. \textbf{Organização Mundial de Saúde declara pandemia do novo
Coronavírus.} Disponível em:
\url{https://www.unasus.gov.br/noticia/organizacao-mundial-de-saude-declara-pandemia-de-coronavirus}.
Acesso em: 10 de jan. de 2023.

VEJA. \textbf{As lições da pandemia de Covid-19 -- que está chegando ao
fim.} 2021 . Disponível em:
\url{https://veja.abril.com.br/saude/os-sinais-de-que-a-pandemia-de-covid-19-vai-acabar-em-breve/}.

\hypertarget{pesquisa-reproduzuxedvel}{%
\subparagraph{Pesquisa Reproduzível}\label{pesquisa-reproduzuxedvel}}

\url{https://www.faac.unesp.br/\#!/pos-graduacao/}

\end{document}
